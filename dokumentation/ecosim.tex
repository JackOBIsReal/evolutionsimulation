\documentclass[12pt]{article}
\usepackage[margin=0.8in]{geometry}
\usepackage[utf8]{inputenc}
\addtolength{\topmargin}{-.175in}
\author{{\Large Jonas Berggren, Jacob Maxton}}
\font\myfont=cmr12 at 30pt
\title{{\myfont Ökosystemsimulation}}

\begin{document}
\maketitle
\tableofcontents
\newpage
\begin{abstract}
In diesem Dokument erklären wir wie wir eine Objektorientierte Simualtione eines Ökosystems entwickelt haben, und was für Erkenntinsse sich daraus ziehen lassen.
Dies ist eine Ojektoirentierte Simulation.
\end{abstract}
\section{Die Simulation}
Die Simulation basiert auf Objektorientierter Programmierung.
Es gibt eine Klasse fúr Tiere und je eine Unterklasse für Füchse und für kaninchen.
Klassen in der Informatik kann man sich vorstellen wie eine Methodenkarte mit derer man beispielsweise ein Kanienchen machen kann.
Darin gespeichert sind Variablen wie Position oder geschlecht, und Methoden.
Dies sind funktionen oder Aktionen die das Kaninchen ausführen kann.
Dies kann z.b. essen, oder Fortpflanzen sein.

Sie haben Instanzvarieblen für Position, Geschlchtecht, Alter, Hunger, Durst, Libido.
\subsection{Methoden}
Bewegung, Forpflanzung, Ernrärung, aussuchen des Ziels, Flucht, 
\section{Die Analyse}




\end{document}
